\section{Grunds"atze}

\subsection{core.logic}

core.logic ist eine Implementierung des auf Scheme basierenden Solvers: miniKanren

Kanren ist Japanisch und bedeutet so viel wie Relation.

miniKanren in einem Satz: Wenn miniKanren ein Ausdruck und eine gew"unschte Ausgabe gegeben wird, kann es dies \dq{}Rückw"arts\dq{} ausf"uhren und findet dabei alle m"oglichen Eingaben zu dem Ausdruck der die gew"unschte Ausgabe erzeugt hat.

\subsubsection{Logische Ausdr"ucke}

Ein logischer Ausdruck ist also eine Anweisung f"ur den Solver und besteht aus den folgenden Teilen:
\begin{itemize}

\item eine Menge von logischen Variablen

\item eine Menge von Beschr"ankungen auf die Werte, die die logischen Variablen annehmen k"onnen

\end{itemize}


\subsection{Logische Variablen}
Logische Variablen sind Container f"ur einen nicht eindeutigen Wert. Das hei\ss{}t, dass eine logische Variable mehrere Werte nacheinander annehmen kann, um diese auszugeben oder weiterzugeben.
%//Wie sind die logischen Variablen implementiert? [...]
Logische Variablen k"onnen spezielle Werte haben, zum Beispiel \_0. Dies soll darstellen, dass die entsprechende logische Variable jeden beliebigen Wert annehmen kann, um die Bedingungen zu erf"ullen.

\begin{lstlisting}
(run * [q r] (== q r))
\end{lstlisting}
\begin{flushleft}
\code{Ausgabe: [\_0 \_0]}
\end{flushleft}
Diese Ausgabe bedeutet, dass beide logischen Variablen q und r jeden beliebigen Wert annehmen können, um die Bedingungen zu erfüllen, dabei müssen sie aber beide den gleichen Wert annehmen.

\begin{lstlisting}
(run * [q r] (== q q) (== r r))
\end{lstlisting}
\begin{flushleft}
\code{Ausgabe: [\_0 \_1]}
\end{flushleft}
Bei dieser Anweisung können q und r auch jeden beliebigen Wert annehmen, dürfen dabei aber auch distinkt voneinander sein, um die Bedingungen zu erfüllen.\\


In core.logic gibt es zwei Wege, um logische Variablen einzuführen:
\begin{itemize}
\item \code{(run * [\dots{}] \dots{})}
\item \code{(fresh [\dots{}] \dots{})}
\end{itemize}

Da \code{(run * [])} einen logischen Ausdruck einleitet, muss hier auch immer mindestens 1 logische Variable eingeführt werden. Weiterhin sind alle logischen Variablen immer nur in dem Bereich und allen tieferen Bereichen verfügbar in denen sie eingeführt wurden.\\
Beispiel:
\begin{lstlisting}
(run * [q] (fresh [x] (== x 1) (== x q)))
\end{lstlisting}
Da die logische Variable x durch \code{fresh} eingeführt wurde, kann diese nur innerhalb des \code{fresh}-Bereichs genutzt werden. Außerhalb der Klammern von \code{fresh} ist x nicht mehr gültig. Die logische Variable q wurde allerdings von  \code{run} eingeführt und ist daher auch innerhalb von \code{fresh} verfügbar und kann dort genutzt werden.


\subsection{Beschr"ankungen}

Beschr"ankungen oder auch Constraints sind Ausdr"ucke die die Werte die eine logische Variable annehmen kann, beschr"anken. Es k"onnen mehrere Beschr"ankungen existieren die untereinander in einer Konjunktion stehen:
\begin{lstlisting}
(run* [q]
    (constraint-1)
    (constraint-2)
    (constraint-3)
)
\end{lstlisting}

Hier muss ein Wert alle 3 Constraints erf"ullen, um als Wert von q angenommen werden zu k"onnen.
\begin{lstlisting}
(run* [q]
  (membero q [1 2 3])
  (membero q [2 3 4]))
\end{lstlisting}

Im Beispiel muss ein Wert in den beiden Mengen \code{[1 2 3]} und \code{[2 3 4]} beinhaltet sein, um von q als Wert angenommen zu werden. Das Ergebnis w"are in diesem Beispiel: \code{[2 3]}.

