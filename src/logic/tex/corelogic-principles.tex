\section{Grunds"atze}

\subsection{core.logic}

core.logic ist eine Implementierung des auf Scheme basierenden Solvers: miniKanren

Kanren ist Japanisch und bedeutet so viel wie Relation.

miniKanren in einem Satz: Wenn miniKanren ein Ausdruck und eine gew"unschte Ausgabe gegeben wird, kann es dies \dq{} R"uckw"arts \dq{} ausf"uhren und findet dabei alle m"oglichen Eingaben zu dem Ausdruck der die gew"unschte Ausgabe erzeugt hat.

\subsubsection{Logische Ausdr"ucke}

Ein logischer Ausdruck ist also eine Anweisung f"ur den Solver und besteht aus den folgenden Teilen:
\begin{itemize}

\item eine Menge von logischen Variablen

\item eine Menge von Beschr"ankungen auf die Werte, die die logischen Variablen annehmen k"onnen

\end{itemize}

Je nach Ausdruck, ist die Anzahl der logischen Variablen, die diese Einf"uhren unterschiedlich. (run ..) zum Beispiel, kann nur eine lvar einf"uhren.


\subsection{Logische Variablen}
Logische Variablen sind Container f"ur einen nicht eindeutigen Wert. Das heiss{}t, dass eine logische Variable mehrere Werte nacheinander annehmen kann, um diese auszugeben oder weiterzugeben.
%//Wie sind die logischen Variablen implementiert? [...]
Logische Variablen k"onnen spezielle Werte haben, zum Beispiel \_0
 
Tafelbild: (run* [q] (== q q)) ; R"uckgabe --> \_0
In diesem Fall wird \_0 zur"uckgegeben, was so viel bedeutet wie: q kann jeden beliebigen Wert annehmen und erf"ullt immer die Beschr"ankung.
F"ur den Fall, dass wir mehrere logische Variablen haben (z.B. 2) und beide k"onnen jeden beliebigen Wert annehmen, m"ussen aber gleich sein ist die R"uckgabe (\_0 \_0), k"onnen beide distinkt voneinander sein: (\_0 \_1).
Logische Variablen werden auf 2 Arten in ein logisches Programm eingef"uhrt:
(run* [q] ...
(fresh [...] ...)
W"ahrend (run* ...) immer nur 1 logische Variable einf"uhren kann, f"uhrt (fresh ...) beliebig viele ein, muss aber innerhalb von einem (run* ...) Befehl stehen.

\subsection{Beschr"ankungen}

Beschr"ankungen oder auch Constraints sind Ausdr"ucke die die Werte die eine logische Variable annehmen kann, beschr"anken. Es k"onnen mehrere Beschr"ankungen existieren die untereinander in einer Konjunktion stehen:
\begin{lstlisting}
(run* [q]
    (constraint-1)
    (constraint-2)
    (constraint-3)
)
\end{lstlisting}

Hier muss ein Wert alle 3 Constraints erf"ullen, um als Wert von q angenommen werden zu k"onnen.
\begin{lstlisting}
(run* [q]
  (membero q [1 2 3])
  (membero q [2 3 4]))
\end{lstlisting}

Im Beispiel muss ein Wert in den beiden Mengen [1 2 3] und [2 3 4] beinhaltet sein, um von q als Wert angenommen zu werden. Ergebnis w"are hier [2 3] .

