\subsection{Grundsätze}

\subsubsection{core.logic}

core.logic ist eine Implementierung des auf Scheme basierenden Solvers: miniKanren

Kanren ist Japanisch und bedeutet so viel wie Relation.

Prinzipiell ist das miniKanren zusammengefasst: Wenn miniKanren ein Ausdruck und eine gewünschte Ausgabe gegeben wird, kann es dies “Rückwärts” ausführen und findet dabei alle möglichen Eingaben zu dem Ausdruck der die gewünschte Ausgabe erzeugt hat.

\subsubsection{Logische Ausdrücke}

Ein logischer Ausdruck ist also eine Anweisung für den Solver und besteht aus den folgenden Teilen:
\begin{itemize}

\item eine Menge von logischen Variablen

\item eine Menge von Beschränkungen auf die Werte die die logischen Variablen annehmen können

\end{itemize}

Je nach Ausdruck, ist die Anzahl der logischen Variablen die diese Einführen unterschiedlich. (run ..) zum Beispiel kann nur 1 lvar einführen.


\subsubsection{Logische Variablen}
Logische Variablen sind Container für einen nicht eindeutigen Wert. Das heißt, dass eine logische Variable mehrere Werte nacheinander annehmen kann, um diese auszugeben oder weiterzugeben.
//Wie sind die logischen Variablen implementiert? [...]
Logische Variablen können spezielle Werte haben, zum Beispiel _0 
Tafelbild: (run* [q] (== q q)) ; Rückgabe --> _0
In diesem Fall wird _0 zurückgegeben, was so viel bedeutet wie: q kann jeden beliebigen Wert annehmen und erfüllt immer die Beschränkung.
Für den Fall, dass wir mehrere logische Variablen haben (zB 2) und beide können jeden beliebigen Wert annehmen, müssen aber gleich sein ist die Rückgabe (_0 _0) können beide distinkt voneinander sein: (_0 _1).
Logische Variablen werden auf 2 Arten in ein logisches Programm eingeführt:
(run* [q] …)
(fresh [...] …)
Während (run* ...) immer nur 1 logische Variable einführen kann, führt (fresh …) beliebig viele ein, muss aber innerhalb von einem (run* …) Befehl stehen.

\subsubsection{Beschränkungen}

Beschränkungen oder auch Constraints sind Ausdrücke die die Werte die eine logische Variable annehmen kann, beschränken. Es können mehrere Beschränkungen existieren die untereinander in einer Konjunktion stehen:
\begin{lstlistings}
(run* [q]
    (constraint-1)
    (constraint-2)
    (constraint-3)
)
\end{lstlistings}
Hier muss ein Wert alle 3 Constraints erfüllen, um als Wert von q angenommen werden zu können.
Im Beispiel am Beamer muss ein Wert in den beiden Mengen \code{[1 2 3]} und \code{[2 3 4]} beinhaltet sein, um von q als Wert angenommen zu werden. Ergebnis wäre hier \code{[2 3]}.

