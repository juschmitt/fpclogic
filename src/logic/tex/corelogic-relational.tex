\subsection{Relationen in der logischen Programmierung}

Funktionen in der logischen Programmierung basieren oft auf Relationen, da diese es erlauben, Bedingungen rückwärts auszuwerten. Eine logische Programmiersprache wie zum Beispiel Prolog, die mit Fakten und Datenbanken arbeitet, erlaubt es in den Fakten, eigene Relationen darzustellen und diese hinterher mit den entsprechenden Funktionen auszuwerten. Darauf gehen wir näher im Kapitel \emph{\nameref{sec:features}} ein.

\subsubsection{Beispiel}

Eine Relation plus stellt eine Abbildung des Kreuzprodukts zweier natürlicher Zahlen auf eine natürliche Zahl dar.

\begin{lstlisting}
plus: N x N -> N
    (a, b) |-> (+ a b)		a   b   plus
				1   1   2
				1   2   3
				2   2   4
\end{lstlisting}
Somit können wir unseren Solver nutzen, um zu prüfen ob eine bestimmte Kombination von Argumenten erlaubt ist.

\begin{lstlisting}
plus_o Relation	 N x N x N
		-(1 1 1)- nicht erlaubt
		 (1 1 2) erlaubt
		 ...
\end{lstlisting}
Relationale Programmierung kann rückwärts ausgewertet werden

\begin{lstlisting}
(run* [q] (== q (plus_o (1 1 q))))
(run* [q] (== q (plus_o (q 1 3))))
(run* [q r] (== q (plus_o q r 3)))
(run* [q r] (== q (plus_o q 1 r)))
(run* [q r s] (== q (plus_o q r s)))
\end{lstlisting}
Deshalb können wir über folgende Relation die Aufgabe mit verschieden umfangreichen Ergebnissen \dq{}lösen\dq{}. Die erste und zweite Zeile ergeben z.B. 2 da \code{1+1=2} und nur 2 mit 1 zusammengenommen 3 ergibt. Zeile 3 hat ein festes Ergebnis und somit ist \code{q} und \code{r} auch begrenzt und es gibt keine unendlich vielen Möglichkeiten. Zeile 4 hat als Ergebnis eine logische Variable und somit gibt es hier unendliche viele Möglichkeiten \code{q} und \code{r} anzuordnen, sodass mit \code{1} eine natürliche Zahl herauskommt. In der letzten Zeile ergibt sich dann auch eine unendliche Zahl an Möglichkeiten, die natürlichen Zahlen zu einer Kombination anzuordnen die als Summe eine natürliche Zahl ergeben.


