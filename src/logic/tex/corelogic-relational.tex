\subsection{Relationale Programmierung}



\subsubsection{Beispiel}

Eine Relation plus stellt eine Abbildung des Kreuzprodukts zweier natürlicher Zahlen auf eine natürliche Zahl dar.

\begin{lstlisting}
plus: N x N -> N
    (a, b) |-> (+ a b)		a   b   plus
				1   1   2
				1   2   3
				2   2   4
\end{lstlisting}

Somit können wir unseren Solver nutzen, um zu prüfen ob eine bestimmte Kombination von Argumenten erlaubt ist.
\begin{lstlisting}
plus_o Relation	 N x N x N
		-(1 1 1)- nicht erlaubt
		 (1 1 2) erlaubt
\end{lstlisting}

Relationale Programmierung kann rückwärts ausgewertet werden

\begin{lstlisting}
(run* [q] (== q (plus_o (1 1 q))))
(run* [q] (== q (plus_o (q 1 3))))
(run* [q r] (== q (plus_o q r 3)))
\end{lstlisting}
