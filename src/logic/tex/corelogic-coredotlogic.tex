\section{core.logic} 

\subsection{Allgemeines zu core.logic}

Die Clojure-Erweiterung core.logic bietet die Möglichkeit, in Clojure, Prolog-ähnlich verschieden Programmierparadigmen zu verfolgen, wie z.B. Relationale Programmierung [...] oder auch logische Constraintprogrammierung. Damit soll eine einfache Möglichkeit geschaffen werden, bei der Lösung von logischen Problemen, die bestehenden Mittel (von core.logic) zu nutzen oder auch zu erweitern.

Momentan stehen David Nolen und Rich Hickey, Erfinder des Lisp-Dialekts Clojure (s. Wikipedia) hinter dem quelloffenen Projekt.

