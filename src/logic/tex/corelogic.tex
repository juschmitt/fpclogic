%==============================================================================
% FPC core.logic
%==============================================================================

\documentclass[twoside,a4paper]{article}

%==============================================================================
%benutzte Packages
%==============================================================================

\usepackage[english, ngerman]{babel}
\usepackage[latin1]{inputenc}
\usepackage{epsf}
\usepackage{rotating}
\usepackage{ifthen}
\usepackage{makeidx}
\usepackage{graphicx}
\usepackage{amsmath}
\usepackage{amssymb}
\usepackage{alltt}
\usepackage{scalefnt}
\usepackage{epsfig}
\usepackage{psfrag}
\usepackage{bbm}
\usepackage{subfigure}
\usepackage{algorithm}
\usepackage{enumerate}
\usepackage{epic}
\usepackage{listings}
\usepackage{hyperref}

\bibliographystyle{abbrv}

%==============================================================================

\setlength{\oddsidemargin}{3.6pt}
\setlength{\evensidemargin}{22.6pt}
\setlength{\textwidth}{426.8pt}
\setlength{\textheight}{654.4pt}
\setlength{\headsep}{18pt}
\setlength{\headheight}{15pt}
\setlength{\topmargin}{-41.7pt}
\setlength{\topskip}{10pt}
\setlength{\footskip}{42pt}
\def\code#1{\texttt{#1}}

%\graphicspath{{}{}}

\title{Logik in Clojure mit core.logic}
\author{Chris Weber und Julian Schmitt}

%==============================================================================
%
% Select the language
%
%==============================================================================

\selectlanguage{ngerman}
%\selectlanguage{english}

%==============================================================================

\begin{document}

\maketitle
\thispagestyle{empty}
\vfill
\tableofcontents
\vfill
\clearpage

%==============================================================================
%
% Insert your LaTeX code below!
%
%==============================================================================

\subsection{Relationale Programmierung}

Eine Relation plus stellt eine Abbildung des Kreuzprodukts zweier nat"urlicher Zahlen auf eine nat"urliche Zahl dar.

\begin{lstlisting}
plus: N x N -> N
    (a, b) |-> (+ a b)		a   b   plus
				1   1   2
				1   2   3
				2   2   4
\end{lstlisting}

Somit k"onnen wir unseren Solver nutzen, um zu pr"ufen ob eine bestimmte Kombination von Argumenten erlaubt ist.
\begin{lstlisting}
plus_o Relation	 N x N x N
		-(1 1 1)- nicht erlaubt
		 (1 1 2) erlaubt
\end{lstlisting}

Relationale Programmierung kann r"uckw"arts ausgewertet werden

\begin{lstlisting}
(run* [q] (== q (plus_o (1 1 q))))
(run* [q] (== q (plus_o (q 1 3))))
(run* [q r] (== q (plus_o q r 3)))
\end{lstlisting}



\subsubsection{Logische Ausdr"ucke}

Ein logischer Ausdruck ist also eine Anweisung f"ur den Solver und besteht aus den folgenden Teilen:
\begin{itemize}

\item eine Menge von logischen Variablen

\item eine Menge von Beschr"ankungen auf die Werte, die die logischen Variablen annehmen k"onnen

\end{itemize}


\subsubsection{Logische Variablen}
Logische Variablen sind Container f"ur einen nicht eindeutigen Wert. Das hei\ss{}t, dass eine logische Variable mehrere Werte nacheinander annehmen kann, um diese auszugeben oder weiterzugeben.
%//Wie sind die logischen Variablen implementiert? [...]
Logische Variablen k"onnen spezielle Werte haben, zum Beispiel \_0. Dies soll darstellen, dass die entsprechende logische Variable jeden beliebigen Wert annehmen kann, um die Bedingungen zu erf"ullen.

\begin{lstlisting}
(run * [q r] (== q r))
\end{lstlisting}
\begin{flushleft}
\code{Ausgabe: [\_0 \_0]}
\end{flushleft}
Diese Ausgabe bedeutet, dass beide logischen Variablen q und r jeden beliebigen Wert annehmen können, um die Bedingungen zu erfüllen, dabei müssen sie aber beide den gleichen Wert annehmen.

\begin{lstlisting}
(run * [q r] (== q q) (== r r))
\end{lstlisting}
\begin{flushleft}
\code{Ausgabe: [\_0 \_1]}
\end{flushleft}
Bei dieser Anweisung können q und r auch jeden beliebigen Wert annehmen, dürfen dabei aber auch distinkt voneinander sein, um die Bedingungen zu erfüllen.\\


In core.logic gibt es zwei Wege, um logische Variablen einzuführen:
\begin{itemize}
\item \code{(run * [\dots{}] \dots{})}
\item \code{(fresh [\dots{}] \dots{})}
\end{itemize}

Da \code{(run * [])} einen logischen Ausdruck einleitet, muss hier auch immer mindestens 1 logische Variable eingeführt werden. Weiterhin sind alle logischen Variablen immer nur in dem Bereich und allen tieferen Bereichen verfügbar in denen sie eingeführt wurden.\\
Beispiel:
\begin{lstlisting}
(run * [q] (fresh [x] (== x 1) (== x q)))
\end{lstlisting}
Da die logische Variable x durch \code{fresh} eingeführt wurde, kann diese nur innerhalb des \code{fresh}-Bereichs genutzt werden. Außerhalb der Klammern von \code{fresh} ist x nicht mehr gültig. Die logische Variable q wurde allerdings von  \code{run} eingeführt und ist daher auch innerhalb von \code{fresh} verfügbar und kann dort genutzt werden.


\subsubsection{Beschr"ankungen}

Beschr"ankungen oder auch Constraints sind Ausdr"ucke die die Werte die eine logische Variable annehmen kann, beschr"anken. Es k"onnen mehrere Beschr"ankungen existieren die untereinander in einer Konjunktion stehen:
\begin{lstlisting}
(run* [q]
    (constraint-1)
    (constraint-2)
    (constraint-3)
)
\end{lstlisting}

Hier muss ein Wert alle 3 Constraints erf"ullen, um als Wert von q angenommen werden zu k"onnen.
\begin{lstlisting}
(run* [q]
  (membero q [1 2 3])
  (membero q [2 3 4]))
\end{lstlisting}

Im Beispiel muss ein Wert in den beiden Mengen \code{[1 2 3]} und \code{[2 3 4]} beinhaltet sein, um von q als Wert angenommen zu werden. Das Ergebnis w"are in diesem Beispiel: \code{[2 3]}.


\section{Einstein-Test oder Zebrapuzzle}

\subsection{Hintergrund}

Bei diesem Rätsel geht es darum, aus einer Menge von 5 Personen, die sich alle jeweils durch die Farbe ihres Hauses, ihr Getränk, ihr Haustier, ihre Zigarettenmarke und ihre Nationalität unterscheiden, mithilfe von gegebenen Informationen und einem logischen Lösungsansatz, genau eine Person mit einer gewissen Eigenschaft herauszufinden. Näheres hierzu z.B. auf Wikipedia.

\subsection{Code}

Das entsprechendes Codebeispiel kann auf folgender Seite \url{https://github.com/swannodette/logic-tutorial#zebras} gefunden werden.


Nachfolgend werden die im Codebeispiel definierten Methoden erklärt.

\subsubsection{righto}

\begin{lstlisting}
(defne righto [x y l]
  ([_ _ [x y . ?r]])
  ([_ _ [_ . ?r]] (righto x y ?r)))
\end{lstlisting}

Diese Methode erzeugt alle Beschränkungen, die wir benötigen, damit \dq{}y\dq{} rechts von \dq{}x\dq{} steht. Ein rekursiver Aufruf sorgt stößt den Prozess sooft wieder an bis der Rest, dargstellt durch das \code{?r} behandelt wurde. Genauer werden also die Constraints zurückgegeben, oder auch Goals, die der Solver benötigt um unser Ergebnis zu errechnen.

\subsubsection{nexto}

\begin{lstlisting}
(defn nexto [x y 1]
  (conde
    ((righto x y 1))
    ((righto y x 1))))
\end{lstlisting}

Diese Methode erzeugt alle Beschränkungen, die wir benötigen, damit \dq{}y\dq{} links oder rechts neben \dq{}x\dq{} steht. Dazu wird die oben erklärte Methode \code{righto} verwendet.

\subsubsection{zebrao}

\begin{lstlisting}
(defn zebrao [hs]
  (macro/symbol-macrolet [_ (lvar)]
    (all
     (== [_ _ [_ _ 'milk _ _] _ _] hs)
     (firsto hs ['norwegian _ _ _ _])
     (nexto ['norwegian _ _ _ _] [_ _ _ _ 'blue] hs)
     (righto [_ _ _ _ 'ivory] [_ _ _ _ 'green] hs)
     (membero ['englishman _ _ _ 'red] hs)
     (membero [_ 'kools _ _ 'yellow] hs)
     (membero ['spaniard _ _ 'dog _] hs)
     (membero [_ _ 'coffee _ 'green] hs)
     (membero ['ukrainian _ 'tea _ _] hs)
     (membero [_ 'lucky-strikes 'oj _ _] hs)
     (membero ['japanese 'parliaments _ _ _] hs)
     (membero [_ 'oldgolds _ 'snails _] hs)
     (nexto [_ _ _ 'horse _] [_ 'kools _ _ _] hs)
     (nexto [_ _ _ 'fox _] [_ 'chesterfields _ _ _] hs))))
\end{lstlisting}

Diese Methode enth"alt s"amtliche Regeln des Puzzles. Durch die Wahl von sprechenden Namen, der verwendeten und selbst definierten Methoden, sind die Regeln sehr gut abzulesen. Die erste Zeile enth"alt z.B. zwei Regeln. Einmal die Regel, das es f"unf H"auser gibt und die Regel, das die Person im mittleren Haus trinkt Milch trinkt. Die zweite Regel sagt aus, das die Person ganz links (firsto, "der Erste") norwegisch ist. Die Dritte, das neben der norwegischen Person ein blaues Haus steht und so weiter.
Im Programmcode werden in der zweiten Zeile die Zeichen \dq{}lvar\dq{} an das Symbol \dq{}\_\dq{} gebunden. Das erspart einige Zeichen Code und erh"oht die Lesbarkeit. 
Die fünf Häuser mit jeweils einer Person und deren fünf verschiedene Eigenschaften, werden intern durch eine 5x5 Matrix dargestellt. Ein Vektor der größe fünf für die Darstellung der Häuser, und jeweils für jedes Haus bzw. die Person und deren Eigenschaften ein Vektor der Größe fünf. Ein \dq{}Haus-Vektor\dq{} hat dabei folgende Bedeutung:

\begin{lstlisting}
["Nationalität" "Zigarettenmarke" "Getränk" "Haustier" "Hausfarbe"]
\end{lstlisting}

\subsection{Lösung des Rätsels}

\begin{lstlisting}
(run 1 [q] (zebrao q))
  ([[norwegian kools _.0 fox yellow]
    [ukrainian chesterfields tea horse blue]
    [englishman oldgolds milk snails red]
    [spaniard lucky-strikes oj dog ivory]
    [japanese parliaments coffee _.1 green]])
\end{lstlisting}

\newpage
%
%==============================================================================
% Bibliography
%==============================================================================
\bibliography{03}

\end{document}

