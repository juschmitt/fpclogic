%==============================================================================
% FPC core.logic
%==============================================================================

\documentclass[twoside,a4paper]{article}

%==============================================================================
%benutzte Packages
%==============================================================================

\usepackage[english, ngerman]{babel}
\usepackage[utf8]{inputenc}
\usepackage{epsf}
\usepackage{rotating}
\usepackage{ifthen}
\usepackage{makeidx}
\usepackage{graphicx}
\usepackage{amsmath}
\usepackage{amssymb}
\usepackage{alltt}
\usepackage{scalefnt}
\usepackage{epsfig}
\usepackage{psfrag}
\usepackage{bbm}
\usepackage{subfigure}
\usepackage{algorithm}
\usepackage{enumerate}
\usepackage{epic}
\usepackage{listings}
\usepackage{hyperref}

\bibliographystyle{abbrv}

%==============================================================================

\setlength{\oddsidemargin}{3.6pt}
\setlength{\evensidemargin}{22.6pt}
\setlength{\textwidth}{426.8pt}
\setlength{\textheight}{654.4pt}
\setlength{\headsep}{18pt}
\setlength{\headheight}{15pt}
\setlength{\topmargin}{-41.7pt}
\setlength{\topskip}{10pt}
\setlength{\footskip}{42pt}
\def\code#1{\texttt{#1}}

%\graphicspath{{}{}}

\title{Logik in Clojure mit core.logic}
\author{Chris Weber und Julian Schmitt}

%==============================================================================
%
% Select the language
%
%==============================================================================

\selectlanguage{ngerman}
%\selectlanguage{english}

%==============================================================================

\begin{document}

\maketitle
\thispagestyle{empty}
\newpage
\thispagestyle{empty}
\vfill
\tableofcontents
\vfill
\clearpage

%==============================================================================
%
% Insert your LaTeX code below!
%
%==============================================================================

\subsection{Grundlegendes}

In diesem Kapitel wird die logische Programmierung kurz vorgestellt, um die Grundlagen auf denen auch das Clojure Paket core.logic besteht vorwegzunehmen. Weiterhin sollen auch Grundz"uge der relationalen Programmierung erkl"art werden, auf der einige Funktionen der logischen Programmierung basieren.

\subsubsection{Grundlagen der logische Programmierung}

Logische Programmierung besteht nicht wie die funktionale Programmierung aus Folgen von Anweisungen, sondern aus Regeln und Fakten mit denen der Interpreter versucht L"osungsaussagen zu treffen. So gibt man zum Beispiel dem Interpreter die Regel, dass die Variable x eine Zahl sein soll, die gleich sein soll mit dem Ergebnis aus \code{2 + 3}.

Der Interpreter oder auch L"osungsmaschine oder Solver genannt, bekommt also ein Ziel (Goal) vorgegeben und versucht dieses mit Hilfe von Fakten R"uckw"arts zu l"osen.


Ein logisches Programm besteht also aus einem oder mehreren Ausdr"ucken und einer L"osungsmaschine. Ein logischer Ausdruck ist ein Ziel, dass die L"osungsmaschine erreichen will.

Ein logischer Ausdruck besteht generell aus einer Menge an logischen Variablen und den entsprechenden Beschr"ankungen auf die Variablen. So stellt aus dem vorherigen Beispiel x die logische Variable dar und \code{x = 2 + 3} ist die Beschr"ankung auf x.


Die wichtigsten Funktionen die eine logische Programmiersprache ausmachen sind die Unifikation, die Einf"uhrung von logischen Variablen und die logische Disjunktion von Beschr"ankungen.

\subsection{Relationale Programmierung}

Eine Relation plus stellt eine Abbildung des Kreuzprodukts zweier nat"urlicher Zahlen auf eine nat"urliche Zahl dar.

\begin{lstlisting}
plus: N x N -> N
    (a, b) |-> (+ a b)		a   b   plus
				1   1   2
				1   2   3
				2   2   4
\end{lstlisting}

Somit k"onnen wir unseren Solver nutzen, um zu pr"ufen ob eine bestimmte Kombination von Argumenten erlaubt ist.
\begin{lstlisting}
plus_o Relation	 N x N x N
		-(1 1 1)- nicht erlaubt
		 (1 1 2) erlaubt
\end{lstlisting}

Relationale Programmierung kann r"uckw"arts ausgewertet werden

\begin{lstlisting}
(run* [q] (== q (plus_o (1 1 q))))
(run* [q] (== q (plus_o (q 1 3))))
(run* [q r] (== q (plus_o q r 3)))
\end{lstlisting}



\subsubsection{Logische Ausdr"ucke}

Ein logischer Ausdruck ist also eine Anweisung f"ur den Solver und besteht aus den folgenden Teilen:
\begin{itemize}

\item eine Menge von logischen Variablen

\item eine Menge von Beschr"ankungen auf die Werte, die die logischen Variablen annehmen k"onnen

\end{itemize}


\subsubsection{Logische Variablen}
Logische Variablen sind Container f"ur einen nicht eindeutigen Wert. Das hei\ss{}t, dass eine logische Variable mehrere Werte nacheinander annehmen kann, um diese auszugeben oder weiterzugeben.
%//Wie sind die logischen Variablen implementiert? [...]
Logische Variablen k"onnen spezielle Werte haben, zum Beispiel \_0. Dies soll darstellen, dass die entsprechende logische Variable jeden beliebigen Wert annehmen kann, um die Bedingungen zu erf"ullen.

\begin{lstlisting}
(run * [q r] (== q r))
\end{lstlisting}
\begin{flushleft}
\code{Ausgabe: [\_0 \_0]}
\end{flushleft}
Diese Ausgabe bedeutet, dass beide logischen Variablen q und r jeden beliebigen Wert annehmen können, um die Bedingungen zu erfüllen, dabei müssen sie aber beide den gleichen Wert annehmen.

\begin{lstlisting}
(run * [q r] (== q q) (== r r))
\end{lstlisting}
\begin{flushleft}
\code{Ausgabe: [\_0 \_1]}
\end{flushleft}
Bei dieser Anweisung können q und r auch jeden beliebigen Wert annehmen, dürfen dabei aber auch distinkt voneinander sein, um die Bedingungen zu erfüllen.\\


In core.logic gibt es zwei Wege, um logische Variablen einzuführen:
\begin{itemize}
\item \code{(run * [\dots{}] \dots{})}
\item \code{(fresh [\dots{}] \dots{})}
\end{itemize}

Da \code{(run * [])} einen logischen Ausdruck einleitet, muss hier auch immer mindestens 1 logische Variable eingeführt werden. Weiterhin sind alle logischen Variablen immer nur in dem Bereich und allen tieferen Bereichen verfügbar in denen sie eingeführt wurden.\\
Beispiel:
\begin{lstlisting}
(run * [q] (fresh [x] (== x 1) (== x q)))
\end{lstlisting}
Da die logische Variable x durch \code{fresh} eingeführt wurde, kann diese nur innerhalb des \code{fresh}-Bereichs genutzt werden. Außerhalb der Klammern von \code{fresh} ist x nicht mehr gültig. Die logische Variable q wurde allerdings von  \code{run} eingeführt und ist daher auch innerhalb von \code{fresh} verfügbar und kann dort genutzt werden.


\subsubsection{Beschr"ankungen}

Beschr"ankungen oder auch Constraints sind Ausdr"ucke die die Werte die eine logische Variable annehmen kann, beschr"anken. Es k"onnen mehrere Beschr"ankungen existieren die untereinander in einer Konjunktion stehen:
\begin{lstlisting}
(run* [q]
    (constraint-1)
    (constraint-2)
    (constraint-3)
)
\end{lstlisting}

Hier muss ein Wert alle 3 Constraints erf"ullen, um als Wert von q angenommen werden zu k"onnen.
\begin{lstlisting}
(run* [q]
  (membero q [1 2 3])
  (membero q [2 3 4]))
\end{lstlisting}

Im Beispiel muss ein Wert in den beiden Mengen \code{[1 2 3]} und \code{[2 3 4]} beinhaltet sein, um von q als Wert angenommen zu werden. Das Ergebnis w"are in diesem Beispiel: \code{[2 3]}.


\subsection{Syntax}

In diesem Kapitel soll die allgemeine Syntax von core.logic, die wichtigsten Funktionen und einige weiterführenden Funktionen vorgestellt werden. Weiterhin werden tiefergreifende Features vorgestellt und erklärt.

\subsubsection{Allgemeine Syntax}

Wie bereits in dem vorhergehenden Kapitel an einigen Beispielen zu sehen war, hat core.logic eine signifikante Syntax.
\begin{lstlisting}
(run * [logic-variables] (logic-expressions in conjunction))
\end{lstlisting}
Dieser Ausdruck liest sich wie folgt: ''Nimm die logischen Ausdrücke, lass den Solver diese lösen und gib alle Werte der logischen Variblen zurück die diese Ausdrücke erfüllen.''\\
\\
Um nicht bei jedem Aufruf der \code{run} Funktion alle Werte der logischen Variable zu bekommen, sondern nur endlich viele, kann man den * nach \code{run} durch eine Zahl ersetzen die der Anzahl der Werte entspricht die zurück gegeben werden sollen.

\subsubsection{Die wichtigsten Funktionen}

core.logic basiert, "ahnlich wie miniKanren, auf 3 grundlegenden Funktionen.
\begin{description}
\item{\code{fresh}:}
Mit fresh lassen sich beliebig viele neue logische Variablen ins Programm einf"uhren. Variablen die durch fresh eingef"uhrt wurden, sind auch nur innerhalb von diesem g"ultig, d.h. lvars innerhalb von fresh m"ussen auf eine au\ss{}erhalb von fresh g"ultige lvar "ubertragen werden.

\item{\code{unify}:}
unify setzt lvars gleich. Entweder zu anderen lvars oder zu Werten. Mit unify lassen sich so zB lvars innerhalb von fresh auf eine lvar au\ss{}erhalb von fresh "ubetragen.

\item{\code{conde}:}
Mit conde ("ahnlich zu cond aus dem clojure.core Paket) lassen sich Constraints so gesagt \dq{}verodern\dq{}. Das heißt es erzeugt eine logische Disjunktion von Constraints.
\end{description}
Beispiel für \code{conde}:
\begin{lstlisting}
(run* [q]
    (conde
        ((unify q 2))
	  *OR*
        ((unify q 1) *AND* (unify q q))
	  *OR*
	((fresh [r s] 
	  (unify r 1)
	    *AND*
	  (unify s 2)
	    *AND*
	  (unify r q)
	    *AND*
	  (unify s q))
)
\end{lstlisting}
Das sind die 3 grundlegenden Funktionen von core.logic. Das gesamte Package beinhaltet aber nat"urlich noch viele mehr, Wie z.B. das eben gesehene (membero \dots{}). Alle weiteren Funktionen im Package bauen aber auf den 3 Basis Funktionen auf. Höhere Funktionen folgen einer bestimmten Namenskonvention, zu sehen z.B. bei conde und memebero. Hiermit werden Funktionen in core.logic die schon im clojure.core existieren mit einem a, e, u oder o \dq{}verlängert\dq{}, um diese von den regulären clojure Funktionen zu differenzieren und diese nicht zu überschreiben. Aber stehen diese Suffixe oft auch für ein bestimmtes Verhalten von Funktionen, sodass der Entiwckler auf den ersten Blick erkennen kann, wie diese Funktion in etwa arbeitet. Diese Namenskonventionen gibt es schon länger, und begründen ihre historische Entstehung in Sprachen wie Prolog und miniKanren.

\begin{itemize}
\item \code{conde}:
Das e steht für \dq{}everyline\dq{} bzw. das jede Zeile von \code{conde} erfolgreich sein, bzw. true zurückgeben kann.

\item \code{conda}:
(Soft cut) Sobald der \dq{}HEAD\dq{} einer Bedingungsanweisung erfolgreich ist, liefert \code{conda} \code{true} zurück und ignoriert alle nachfolgenden Anweisungen. \code{conda} ist nicht-relational.

\item \code{condu}:
(Commited choice) Soblad der \dq{}HEAD\dq{} einer Bedingungsanweisung erfolgreich ist, werden die verbleibenden \dq{}goals\dq{} der Anweisung nur einmal ausgeführt. \code{condu} ist nicht-relational.

\item \code{membero}, \code{anyo}:
Das \dq{}o\dq{} bedeutet das hier eine Relation behandelt wird.
\end{itemize}


\subsubsection{Aufruf des Solvers}

Einfaches Beispiel:

\begin{lstlisting}
( run 1 [q] 
	(== 1 q)
)
\end{lstlisting}
\begin{description}
\item{\code{run 1}:}
Mit \code{run} wird der Solver gestartet und dieser soll das erste Ergebnis, das er bekommt, zurückgeben.
\item{\code{[q]}:}
Das ist die logische Variable für die der Solver Werte suchen soll.
\item{\code{(== 1 q)}:}
Das ist die Beschränkung auf die logische Variable. q wird hier mit 1 unifiziert und gibt damit vor, dass q = 1 sein muss damit diese Beschränkung erfüllt ist.
\end{description}
Werden mehrere Beschränkungen definiert, macht es für den Solver keinen Unterschied in welcher Reihenfolge diese stehen. 

\subsubsection{Höhere Funktionen}

Neben den Funktionen \code{unify (==), fresh und conde} verfügt das Paket core.logic um einige weitere Funktionen die auf diesen 3 grundlegenden Funktionen aufbauen.

Dazu gehört zum Beispiel das bereits genannte \code{membero}:\\
\code{(membero x M)} beschränkt die logische Variable (in diesem Fall x) so, dass diese ein Element der Menge M sein muss, damit die Beschränkung erfüllt ist.\\
\begin{lstlisting}
(run * [q]
	(membero q [1 2 3])
)
\end{lstlisting}
Dieses Beispiel würde die Ausgabe \code{(1, 2, 3)} zurückgeben, da q eine dieser 3 Zahlen annehmen kann, um ein Element der Menge \code{[1 2 3]} zu sein.

\begin{lstlisting}
(defne membero
	[x l]
	([_ [x . tail]])
	([_ [head . tail]]
	 (membero x tail)
	)
)
\end{lstlisting}
Die Definition von \code{membero} in core.logic. Diese besteht aus zwei Beschränkungen, \code{([\_ [x . tail]])} und \code{([\_ [head . tail]] (membero x tail))}. Während die erste Beschränkung erfüllt ist, wenn x das erste Element der Menge l ist, besagt die zweite, dass wenn x nicht das erste Element der Menge ist, dann ist x das erste Element der Menge tail (wobei tail die Menge l ohne deren erstes Element darstellt).

Weitere höhere Funktionen sind:
\begin{description}
\item{\code{(resto l r)}:}
\code{resto} schränkt die logische Variable so ein, dass r die Restmenge der Menge l ist. Das heißt r ist die Menge l ohne deren erstes Element.\\
\\
Die Implementierung der Funktion \code{resto}:
\begin{lstlisting}
(defn resto 
	[l d]
	(fresh [a]
	  (== (lcons a d) l))
)
\end{lstlisting}
\code{resto} werden zwei Variablen übergeben: l und d - wobei l die Gesamtmenge und d die Restmenge darstellt. Weiterhin wird aber noch der Kopf der Menge benötigt, welche mit \code{fresh [a]} eingeführt wird.
Die Funktion \code{(lcons a d)} macht nichts anderes als aus den beiden Variablen a und d eine ordentliche Menge zu erstellen, mit a als Kopf und d als Restmenge. Diese soll dann der Menge l entsprechen und wird daher mit dieser unifiziert.\\
\\
Beispiel zur Funktionsweise von \code{resto}:
\begin{lstlisting}
(run * [q]
	(resto [1 2 3 4] q)
)
\end{lstlisting}
q entspricht in diesem Beispiel der Menge (2 3 4).
\item{\code{(conso x r s)}:}
\code{conso} schränkt die logische Variable so ein, dass x das erste Element einer Menge, r der Rest dieser Menge und s genau diese Menge ist.
\begin{lstlisting}
(defn conso
	[a d l]
	(== (lcons a d) l)
)
\end{lstlisting}
\code{conso} und \code{resto} arbeiten auf die gleiche Art und Weise, nur, dass \code{conso} drei Variablen übergeben bekommt (Kopf, Rest- und Gesamtmenge). \code{(lcons a d)} erstellt wieder aus Kopf und Restmenge eine komplette Menge, die dann mit der übergebenen Gesamtmenge unifiziert wird.\\
\\
Beispiele zur Funktionsweise von \code{conso}:
\begin{lstlisting}
(run * [q]
	(conso 1 [2 3 4] q)
)
\end{lstlisting}
q entspricht hier der Menge (1 2 3 4)

\begin{lstlisting}
(run * [q]
	(conso q [2 3 4] [1 2 3 4])
)
\end{lstlisting}
q entspricht hier 1
\end{description}

\section{Einstein-Test oder Zebrapuzzle}

\subsection{Hintergrund}

Bei diesem Rätsel geht es darum, aus einer Menge von 5 Personen, die sich alle jeweils durch die Farbe ihres Hauses, ihr Getränk, ihr Haustier, ihre Zigarettenmarke und ihre Nationalität unterscheiden, mithilfe von gegebenen Informationen und einem logischen Lösungsansatz, genau eine Person mit einer gewissen Eigenschaft herauszufinden. Näheres hierzu z.B. auf Wikipedia.

\subsection{Code}

Das entsprechendes Codebeispiel kann auf folgender Seite \url{https://github.com/swannodette/logic-tutorial#zebras} gefunden werden.


Nachfolgend werden die im Codebeispiel definierten Methoden erklärt.

\subsubsection{righto}

\begin{lstlisting}
(defne righto [x y l]
  ([_ _ [x y . ?r]])
  ([_ _ [_ . ?r]] (righto x y ?r)))
\end{lstlisting}

Diese Methode erzeugt alle Beschränkungen, die wir benötigen, damit \dq{}y\dq{} rechts von \dq{}x\dq{} steht. Ein rekursiver Aufruf sorgt stößt den Prozess sooft wieder an bis der Rest, dargstellt durch das \code{?r} behandelt wurde. Genauer werden also die Constraints zurückgegeben, oder auch Goals, die der Solver benötigt um unser Ergebnis zu errechnen.

\subsubsection{nexto}

\begin{lstlisting}
(defn nexto [x y 1]
  (conde
    ((righto x y 1))
    ((righto y x 1))))
\end{lstlisting}

Diese Methode erzeugt alle Beschränkungen, die wir benötigen, damit \dq{}y\dq{} links oder rechts neben \dq{}x\dq{} steht. Dazu wird die oben erklärte Methode \code{righto} verwendet.

\subsubsection{zebrao}

\begin{lstlisting}
(defn zebrao [hs]
  (macro/symbol-macrolet [_ (lvar)]
    (all
     (== [_ _ [_ _ 'milk _ _] _ _] hs)
     (firsto hs ['norwegian _ _ _ _])
     (nexto ['norwegian _ _ _ _] [_ _ _ _ 'blue] hs)
     (righto [_ _ _ _ 'ivory] [_ _ _ _ 'green] hs)
     (membero ['englishman _ _ _ 'red] hs)
     (membero [_ 'kools _ _ 'yellow] hs)
     (membero ['spaniard _ _ 'dog _] hs)
     (membero [_ _ 'coffee _ 'green] hs)
     (membero ['ukrainian _ 'tea _ _] hs)
     (membero [_ 'lucky-strikes 'oj _ _] hs)
     (membero ['japanese 'parliaments _ _ _] hs)
     (membero [_ 'oldgolds _ 'snails _] hs)
     (nexto [_ _ _ 'horse _] [_ 'kools _ _ _] hs)
     (nexto [_ _ _ 'fox _] [_ 'chesterfields _ _ _] hs))))
\end{lstlisting}

Diese Methode enth"alt s"amtliche Regeln des Puzzles. Durch die Wahl von sprechenden Namen, der verwendeten und selbst definierten Methoden, sind die Regeln sehr gut abzulesen. Die erste Zeile enth"alt z.B. zwei Regeln. Einmal die Regel, das es f"unf H"auser gibt und die Regel, das die Person im mittleren Haus trinkt Milch trinkt. Die zweite Regel sagt aus, das die Person ganz links (firsto, "der Erste") norwegisch ist. Die Dritte, das neben der norwegischen Person ein blaues Haus steht und so weiter.
Im Programmcode werden in der zweiten Zeile die Zeichen \dq{}lvar\dq{} an das Symbol \dq{}\_\dq{} gebunden. Das erspart einige Zeichen Code und erh"oht die Lesbarkeit. 
Die fünf Häuser mit jeweils einer Person und deren fünf verschiedene Eigenschaften, werden intern durch eine 5x5 Matrix dargestellt. Ein Vektor der größe fünf für die Darstellung der Häuser, und jeweils für jedes Haus bzw. die Person und deren Eigenschaften ein Vektor der Größe fünf. Ein \dq{}Haus-Vektor\dq{} hat dabei folgende Bedeutung:

\begin{lstlisting}
["Nationalität" "Zigarettenmarke" "Getränk" "Haustier" "Hausfarbe"]
\end{lstlisting}

\subsection{Lösung des Rätsels}

\begin{lstlisting}
(run 1 [q] (zebrao q))
  ([[norwegian kools _.0 fox yellow]
    [ukrainian chesterfields tea horse blue]
    [englishman oldgolds milk snails red]
    [spaniard lucky-strikes oj dog ivory]
    [japanese parliaments coffee _.1 green]])
\end{lstlisting}

\newpage
%
%==============================================================================
% Bibliography
%==============================================================================
\bibliography{03}

\end{document}

