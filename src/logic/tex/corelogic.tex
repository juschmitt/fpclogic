%==============================================================================
% FPC core.logic
%==============================================================================

\documentclass[twoside,a4paper]{article}

%==============================================================================
%benutzte Packages
%==============================================================================

\usepackage[english, ngerman]{babel}
\usepackage[utf8]{inputenc}
\usepackage{epsf}
\usepackage{rotating}
\usepackage{ifthen}
\usepackage{makeidx}
\usepackage{graphicx}
\usepackage{amsmath}
\usepackage{amssymb}
\usepackage{alltt}
\usepackage{scalefnt}
\usepackage{epsfig}
\usepackage{psfrag}
\usepackage{bbm}
\usepackage{subfigure}
\usepackage{algorithm}
\usepackage{enumerate}
\usepackage{epic}
\usepackage{listings}
\usepackage{hyperref}

\bibliographystyle{abbrv}

%==============================================================================

\setlength{\oddsidemargin}{3.6pt}
\setlength{\evensidemargin}{22.6pt}
\setlength{\textwidth}{426.8pt}
\setlength{\textheight}{654.4pt}
\setlength{\headsep}{18pt}
\setlength{\headheight}{15pt}
\setlength{\topmargin}{-41.7pt}
\setlength{\topskip}{10pt}
\setlength{\footskip}{42pt}
\def\code#1{\texttt{#1}}

%\graphicspath{{}{}}

\title{Logik in Clojure mit core.logic}
\author{Chris Weber und Julian Schmitt}

%==============================================================================
%
% Select the language
%
%==============================================================================

\selectlanguage{ngerman}
%\selectlanguage{english}

%==============================================================================

\begin{document}

\maketitle
\thispagestyle{empty}
\newpage
\thispagestyle{empty}
\vfill
\tableofcontents
\vfill
\clearpage

%==============================================================================
%
% Insert your LaTeX code below!
%
%==============================================================================

\section{Einf"uhrung}

\subsection{Logische Programmierung}

Logische Programmierung besteht nicht wie die Funktionale Programmierung aus Folgen von Anweisungen, sondern aus Regeln und Fakten mit denen der Interpreter versucht L"osungsaussagen zu treffen. So gibt man zum Beispiel dem Interpreter die Regel, dass die Variable x eine Zahl sein soll, die gleich sein soll mit dem Ergebnis aus \code{2 + 3}.

Der Interpreter oder auch L"osungsmaschine oder Solver genannt, bekommt also ein Ziel (Goal) vorgegeben und versucht dieses mit Hilfe von Fakten R"uckw"arts zu l"osen.

\subsection{Aufbau eines logischen Programms}

Ein logisches Programm besteht also aus einem oder mehreren Ausdr"ucken und einer L"osungsmaschine. Ein logischer Ausdruck ist ein Ziel, dass die L"osungsmaschine erreichen will.

Ein logischer Ausdruck besteht generell aus einer Menge an logischen Variablen und den entsprechenden Beschr"ankungen auf die Variablen. So stellt aus dem vorherigen Beispiel x die logische Variable dar und \code{x = 2 + 3} ist die Beschr"ankung auf x.

\subsection{Relationen in der logischen Programmierung}

Funktionen in der logischen Programmierung basieren oft auf Relationen, da diese es erlauben, Bedingungen rückwärts auszuwerten. Eine logische Programmiersprache wie zum Beispiel Prolog, die mit Fakten und Datenbanken arbeitet, erlaubt es in den Fakten, eigene Relationen darzustellen und diese hinterher mit den entsprechenden Funktionen auszuwerten. Darauf gehen wir näher im Kapitel \emph{\nameref{sec:features}} ein.

\subsubsection{Beispiel}

Eine Relation plus stellt eine Abbildung des Kreuzprodukts zweier natürlicher Zahlen auf eine natürliche Zahl dar.

\begin{lstlisting}
plus: N x N -> N
    (a, b) |-> (+ a b)		a   b   plus
				1   1   2
				1   2   3
				2   2   4
\end{lstlisting}
Somit können wir unseren Solver nutzen, um zu prüfen ob eine bestimmte Kombination von Argumenten erlaubt ist.

\begin{lstlisting}
plus_o Relation	 N x N x N
		-(1 1 1)- nicht erlaubt
		 (1 1 2) erlaubt
		 ...
\end{lstlisting}
Relationale Programmierung kann rückwärts ausgewertet werden

\begin{lstlisting}
(run* [q] (== q (plus_o (1 1 q))))
(run* [q] (== q (plus_o (q 1 3))))
(run* [q r] (== q (plus_o q r 3)))
(run* [q r] (== q (plus_o q 1 r)))
(run* [q r s] (== q (plus_o q r s)))
\end{lstlisting}
Deshalb können wir über folgende Relation die Aufgabe mit verschieden umfangreichen Ergebnissen \dq{}lösen\dq{}. Die erste und zweite Zeile ergeben z.B. 2 da \code{1+1=2} und nur 2 mit 1 zusammengenommen 3 ergibt. Zeile 3 hat ein festes Ergebnis und somit ist \code{q} und \code{r} auch begrenzt und es gibt keine unendlich vielen Möglichkeiten. Zeile 4 hat als Ergebnis eine logische Variable und somit gibt es hier unendliche viele Möglichkeiten \code{q} und \code{r} anzuordnen, sodass mit \code{1} eine natürliche Zahl herauskommt. In der letzten Zeile ergibt sich dann auch eine unendliche Zahl an Möglichkeiten, die natürlichen Zahlen zu einer Kombination anzuordnen die als Summe eine natürliche Zahl ergeben.



\section{core.logic} 

\subsection{Allgemeines}

Die Clojure-Erweiterung core.logic bietet die Möglichkeit, in Clojure, Prolog-ähnlich verschieden Programmierparadigmen zu verfolgen, wie z.B. Relationale Programmierung [...] oder auch logische Constraintprogrammierung. Damit soll eine einfache Möglichkeit geschaffen werden, bei der Lösung von logischen Problemen, die bestehenden Mittel (von core.logic) zu nutzen oder auch zu erweitern.

Momentan stehen David Nolen und Rich Hickey, Erfinder des Lisp-Dialekts Clojure (s. Wikipedia) hinter dem quelloffenen Projekt.

\subsection{}
\section{Grunds"atze}

\subsection{core.logic}

core.logic ist eine Implementierung des auf Scheme basierenden Solvers: miniKanren

Kanren ist Japanisch und bedeutet so viel wie Relation.

miniKanren in einem Satz: Wenn miniKanren ein Ausdruck und eine gew"unschte Ausgabe gegeben wird, kann es dies \dq{} R"uckw"arts \dq{} ausf"uhren und findet dabei alle m"oglichen Eingaben zu dem Ausdruck der die gew"unschte Ausgabe erzeugt hat.

\subsubsection{Logische Ausdr"ucke}

Ein logischer Ausdruck ist also eine Anweisung f"ur den Solver und besteht aus den folgenden Teilen:
\begin{itemize}

\item eine Menge von logischen Variablen

\item eine Menge von Beschr"ankungen auf die Werte, die die logischen Variablen annehmen k"onnen

\end{itemize}

Je nach Ausdruck, ist die Anzahl der logischen Variablen, die diese Einf"uhren unterschiedlich. (run ..) zum Beispiel, kann nur eine lvar einf"uhren.


\subsection{Logische Variablen}
Logische Variablen sind Container f"ur einen nicht eindeutigen Wert. Das heiss{}t, dass eine logische Variable mehrere Werte nacheinander annehmen kann, um diese auszugeben oder weiterzugeben.
%//Wie sind die logischen Variablen implementiert? [...]
Logische Variablen k"onnen spezielle Werte haben, zum Beispiel \_0
 
Tafelbild: (run* [q] (== q q)) ; R"uckgabe --> \_0
In diesem Fall wird \_0 zur"uckgegeben, was so viel bedeutet wie: q kann jeden beliebigen Wert annehmen und erf"ullt immer die Beschr"ankung.
F"ur den Fall, dass wir mehrere logische Variablen haben (z.B. 2) und beide k"onnen jeden beliebigen Wert annehmen, m"ussen aber gleich sein ist die R"uckgabe (\_0 \_0), k"onnen beide distinkt voneinander sein: (\_0 \_1).
Logische Variablen werden auf 2 Arten in ein logisches Programm eingef"uhrt:
(run* [q] ...
(fresh [...] ...)
W"ahrend (run* ...) immer nur 1 logische Variable einf"uhren kann, f"uhrt (fresh ...) beliebig viele ein, muss aber innerhalb von einem (run* ...) Befehl stehen.

\subsection{Beschr"ankungen}

Beschr"ankungen oder auch Constraints sind Ausdr"ucke die die Werte die eine logische Variable annehmen kann, beschr"anken. Es k"onnen mehrere Beschr"ankungen existieren die untereinander in einer Konjunktion stehen:
\begin{lstlisting}
(run* [q]
    (constraint-1)
    (constraint-2)
    (constraint-3)
)
\end{lstlisting}

Hier muss ein Wert alle 3 Constraints erf"ullen, um als Wert von q angenommen werden zu k"onnen.
\begin{lstlisting}
(run* [q]
  (membero q [1 2 3])
  (membero q [2 3 4]))
\end{lstlisting}

Im Beispiel muss ein Wert in den beiden Mengen [1 2 3] und [2 3 4] beinhaltet sein, um von q als Wert angenommen zu werden. Ergebnis w"are hier [2 3] .


\subsection{Syntax}

In diesem Kapitel soll die allgemeine Syntax von core.logic, die wichtigsten Funktionen und einige weiterführenden Funktionen vorgestellt werden. Weiterhin werden tiefergreifende Features vorgestellt und erklärt.

\subsubsection{Allgemeine Syntax}

Wie bereits in dem vorhergehenden Kapitel an einigen Beispielen zu sehen war, hat core.logic eine signifikante Syntax.
\begin{lstlisting}
(run * [logic-variables] (logic-expressions in conjunction))
\end{lstlisting}
Dieser Ausdruck liest sich wie folgt: ''Nimm die logischen Ausdrücke, lass den Solver diese lösen und gib alle Werte der logischen Variblen zurück die diese Ausdrücke erfüllen.''\\
\\
Um nicht bei jedem Aufruf der \code{run} Funktion alle Werte der logischen Variable zu bekommen, sondern nur endlich viele, kann man den * nach \code{run} durch eine Zahl ersetzen die der Anzahl der Werte entspricht die zurück gegeben werden sollen.

\subsubsection{Die wichtigsten Funktionen}

core.logic basiert, "ahnlich wie miniKanren, auf 3 grundlegenden Funktionen.
\begin{description}
\item{\code{fresh}:}
Mit fresh lassen sich beliebig viele neue logische Variablen ins Programm einf"uhren. Variablen die durch fresh eingef"uhrt wurden, sind auch nur innerhalb von diesem g"ultig, d.h. lvars innerhalb von fresh m"ussen auf eine au\ss{}erhalb von fresh g"ultige lvar "ubertragen werden.

\item{\code{unify}:}
unify setzt lvars gleich. Entweder zu anderen lvars oder zu Werten. Mit unify lassen sich so zB lvars innerhalb von fresh auf eine lvar au\ss{}erhalb von fresh "ubetragen.

\item{\code{conde}:}
Mit conde ("ahnlich zu cond aus dem clojure.core Paket) lassen sich Constraints so gesagt \dq{}verodern\dq{}. Das heißt es erzeugt eine logische Disjunktion von Constraints.
\end{description}
Beispiel für \code{conde}:
\begin{lstlisting}
(run* [q]
    (conde
        ((unify q 2))
	  *OR*
        ((unify q 1) *AND* (unify q q))
	  *OR*
	((fresh [r s] 
	  (unify r 1)
	    *AND*
	  (unify s 2)
	    *AND*
	  (unify r q)
	    *AND*
	  (unify s q))
)
\end{lstlisting}
Das sind die 3 grundlegenden Funktionen von core.logic. Das gesamte Package beinhaltet aber nat"urlich noch viele mehr, Wie z.B. das eben gesehene (membero \dots{}). Alle weiteren Funktionen im Package bauen aber auf den 3 Basis Funktionen auf. H"ohere Funktionen, folgen einer bestimmten Namenskonvention, wie z.B. zu sehen bei conde und memebero, werden Funktionen in core.logic die schon im clojure.core existieren mit einem a, e, u oder o um diese von den regul"aren clojure Funktionen zu differenzieren und diese nicht zu "uberschreiben. Weiterhin stehen diese Suffixe oft auch für bestimmte Arten von Funktionen, sodass der Entiwckler auf den ersten Blick erkennen kann, wie diese Funktion in etwa arbeitet. Ansonsten bestehen diese Namenskonventionen hauptsächlich aus historischen Gründen.


\subsubsection{Aufruf des Solvers}

Einfaches Beispiel:

\begin{lstlisting}
( run 1 [q] 
	(== 1 q)
)
\end{lstlisting}
\begin{description}
\item{\code{run 1}:}
Mit \code{run} wird der Solver gestartet und dieser soll das erste Ergebnis, das er bekommt, zurückgeben.
\item{\code{[q]}:}
Das ist die logische Variable für die der Solver Werte suchen soll.
\item{\code{(== 1 q)}:}
Das ist die Beschränkung auf die logische Variable. q wird hier mit 1 unifiziert und gibt damit vor, dass q = 1 sein muss damit diese Beschränkung erfüllt ist.
\end{description}
Werden mehrere Beschränkungen definiert, macht es für den Solver keinen Unterschied in welcher Reihenfolge diese stehen. 

\subsubsection{Höhere Funktionen}

Neben den Funktionen \code{unify (==), fresh und conde} verfügt das Paket core.logic um einige weitere Funktionen die auf diesen 3 grundlegenden Funktionen aufbauen.

Dazu gehört zum Beispiel das bereits genannte \code{membero}:\\
\code{(membero x M)} beschränkt die logische Variable (in diesem Fall x) so, dass diese ein Element der Menge M sein muss, damit die Beschränkung erfüllt ist.\\
\begin{lstlisting}
(run * [q]
	(membero q [1 2 3])
)
\end{lstlisting}
Dieses Beispiel würde die Ausgabe \code{(1, 2, 3)} zurückgeben, da q eine dieser 3 Zahlen annehmen kann, um ein Element der Menge \code{[1 2 3]} zu sein.

\begin{lstlisting}
(defne membero
	[x l]
	([_ [x . tail]])
	([_ [head . tail]]
	 (membero x tail)
	)
)
\end{lstlisting}
Die Definition von \code{membero} in core.logic. Diese besteht aus zwei Beschränkungen, \code{([\_ [x . tail]])} und \code{([\_ [head . tail]] (membero x tail))}. Während die erste Beschränkung erfüllt ist, wenn x das erste Element der Menge l ist, besagt die zweite, dass wenn x nicht das erste Element der Menge ist, dann ist x das erste Element der Menge tail (wobei tail die Menge l ohne deren erstes Element darstellt).

Weitere höhere Funktionen sind:
\begin{description}
\item{\code{(resto l r)}:}
\code{resto} schränkt die logische Variable so ein, dass r die Restmenge der Menge l ist. Das heißt r ist die Menge l ohne deren erstes Element.\\
\\
Die Implementierung der Funktion \code{resto}:
\begin{lstlisting}
(defn resto 
	[l d]
	(fresh [a]
	  (== (lcons a d) l))
)
\end{lstlisting}
\code{resto} werden zwei Variablen übergeben: l und d - wobei l die Gesamtmenge und d die Restmenge darstellt. Weiterhin wird aber noch der Kopf der Menge benötigt, welche mit \code{fresh [a]} eingeführt wird.
Die Funktion \code{(lcons a d)} macht nichts anderes als aus den beiden Variablen a und d eine ordentliche Menge zu erstellen, mit a als Kopf und d als Restmenge. Diese soll dann der Menge l entsprechen und wird daher mit dieser unifiziert.\\
\\
Beispiel zur Funktionsweise von \code{resto}:
\begin{lstlisting}
(run * [q]
	(resto [1 2 3 4] q)
)
\end{lstlisting}
q entspricht in diesem Beispiel der Menge (2 3 4).
\item{\code{(conso x r s)}:}
\code{conso} schränkt die logische Variable so ein, dass x das erste Element einer Menge, r der Rest dieser Menge und s genau diese Menge ist.
\begin{lstlisting}
(defn conso
	[a d l]
	(== (lcons a d) l)
)
\end{lstlisting}
\code{conso} und \code{resto} arbeiten auf die gleiche Art und Weise, nur, dass \code{conso} drei Variablen übergeben bekommt (Kopf, Rest- und Gesamtmenge). \code{(lcons a d)} erstellt wieder aus Kopf und Restmenge eine komplette Menge, die dann mit der übergebenen Gesamtmenge unifiziert wird.\\
\\
Beispiele zur Funktionsweise von \code{conso}:
\begin{lstlisting}
(run * [q]
	(conso 1 [2 3 4] q)
)
\end{lstlisting}
q entspricht hier der Menge (1 2 3 4)

\begin{lstlisting}
(run * [q]
	(conso q [2 3 4] [1 2 3 4])
)
\end{lstlisting}
q entspricht hier 1
\end{description}

\subsection{Einstein-Test oder Zebrapuzzle}

Bei diesem R"atsel geht es darum, aus einer Menge von 5 Personen, die sich alles jeweils durch die Farbe ihres Hauses, ihr Getr"ank, Haustier, ihre Zigarettenmarke und ihrer Nationalit"at unterscheiden, mithilfe von unvollst"andiger Informationen und einem logischen L"osungsansatz, genau eine Person mit einer gewissen Eigenschaft herauszufinden. Näheres z.B. auf Wikipedia.

\subsubsection{Code}

Das entsprechendes Codebeispiel kann auf folgender Seite \url{https://github.com/swannodette/logic-tutorial#zebras} gefunden werden.

Die Methode \code{righto}

\begin{lstlisting}
(defne righto [x y l]
  ([_ _ [x y . ?r]])
  ([_ _ [_ . ?r]] (righto x y ?r)))
\end{lstlisting}

erzeugt Pr"adikate um alle Permutationen eines Objekts zu bekommen, wenn es sich auf der rechten Seite befindet. Ein rekursiver Aufruf sorgt für die Ausgabe aller geforderten Pr"adikate.

Die Methode \code{nexto}

\begin{lstlisting}
(defn nexto [x y 1]
  (conde
    ((righto x y 1))
    ((righto y x 1))))
\end{lstlisting}

erzeugt Pr"adikate um alle Permutationen eines Objektes zu bekommen, wenn es sich daneben befindet.

Die Methode \code{zebrao}

\begin{lstlisting}
(defn zebrao [hs]
  (macro/symbol-macrolet [_ (lvar)]
    (all
     (== [_ _ [_ _ 'milk _ _] _ _] hs)
     (firsto hs ['norwegian _ _ _ _])
     (nexto ['norwegian _ _ _ _] [_ _ _ _ 'blue] hs)
     (righto [_ _ _ _ 'ivory] [_ _ _ _ 'green] hs)
     (membero ['englishman _ _ _ 'red] hs)
     (membero [_ 'kools _ _ 'yellow] hs)
     (membero ['spaniard _ _ 'dog _] hs)
     (membero [_ _ 'coffee _ 'green] hs)
     (membero ['ukrainian _ 'tea _ _] hs)
     (membero [_ 'lucky-strikes 'oj _ _] hs)
     (membero ['japanese 'parliaments _ _ _] hs)
     (membero [_ 'oldgolds _ 'snails _] hs)
     (nexto [_ _ _ 'horse _] [_ 'kools _ _ _] hs)
     (nexto [_ _ _ 'fox _] [_ 'chesterfields _ _ _] hs))))
\end{lstlisting}

enth"alt s"amtliche Regeln in Pr"adikatenform. Die erste Zeile enth"alt z.B. zwei Regeln. Einmal die Regel, das es f"unf H"auser gibt und die Regel, das die Person im mittleren Haus trinkt Milch trinkt. Die zweite Regel sagt aus, das die Person ganz links (firsto, "der Erste") norwegisch ist. Die Dritte, das neben der norwegischen Person ein blaues Haus steht und so weiter.
Im Programmcode werden in der zweiten Zeile die Zeichen "lvar" an das Symbol "_" gebunden. Das erspart einige Zeichen Code und erhöt die Lesbarkeit.


\newpage
%
%==============================================================================
% Bibliography
%==============================================================================
\bibliography{03}

\end{document}

