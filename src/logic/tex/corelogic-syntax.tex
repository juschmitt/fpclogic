\subsection{Syntax}

Da in Clojure umgesetzt, findet sich jeder Eingeweite schnell zurecht. Ähnlich zu Scheme?

\subsection{Aufruf der Instanz eines Lösungsautomaten oder auch Solver}

\begin{lstlistings}
    (run* [logic-variable] &constraints)
\end{lstlistings}


Tafelbild: (run* [q] (== 1 q) ) ; Rückgabe --> 1
(run* ...) -> Befehl für den Solver; * ist Anzahl der Ergebnisse, kann auch entsprechend eine Zahl sein.
[q] -> ist die logische Variable für die ein oder mehrere Werte gesucht wird
(== 1 q) -> ist die Beschränkung für q.
“Gib alle Werte für q zurück für die gilt: q == 1”


\subsection{Logische Ausdrücke}

Ein logischer Ausdruck ist also eine Anweisung für den Solver und besteht aus den folgenden Teilen:
\begin{itemize}

\item eine Menge von logischen Variablen

\item eine Menge von Beschränkungen auf die Werte die die logischen Variablen annehmen können

\end{itemize}

Je nach Ausdruck, ist die Anzahl der logischen Variablen die diese Einführen unterschiedlich. (run ..) zum Beispiel kann nur 1 lvar einführen.
