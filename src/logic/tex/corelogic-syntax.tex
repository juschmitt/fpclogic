\subsection{Syntax}

In diesem Kapitel soll die allgemeine Syntax von core.logic, die wichtigsten Funktionen und einige weiterführenden Funktionen vorgestellt werden. Weiterhin werden tiefergreifende Features vorgestellt und erklärt.

\subsubsection{Allgemeine Syntax}

Wie bereits in dem vorhergehenden Kapitel an einigen Beispielen zu sehen war, hat core.logic eine signifikante Syntax.
\begin{lstlisting}
(run * [logic-variables] (logic-expressions in conjunction))
\end{lstlisting}
Dieser Ausdruck liest sich wie folgt: ''Nimm die logischen Ausdrücke, lass den Solver diese lösen und gib alle Werte der logischen Variblen zurück die diese Ausdrücke erfüllen.''\\

Um nicht bei jedem Aufruf der \code{run} Funktion alle Werte der logischen Variable zu bekommen, sondern nur endlich viele, kann man den * nach \code{run} durch eine Zahl ersetzen die der Anzahl der Werte entspricht die zurück gegeben werden sollen.

\subsubsection{Die wichtigsten Funktionen}

core.logic basiert, "ahnlich wie miniKanren, auf 3 grundlegenden Funktionen.
\begin{description}
\item{\code{fresh}:}
Mit fresh lassen sich beliebig viele neue logische Variablen ins Programm einf"uhren. Variablen die durch fresh eingef"uhrt wurden, sind auch nur innerhalb von diesem g"ultig, d.h. lvars innerhalb von fresh m"ussen auf eine au\ss{}erhalb von fresh g"ultige lvar "ubertragen werden.

\item{\code{unify}:}
unify setzt lvars gleich. Entweder zu anderen lvars oder zu Werten. Mit unify lassen sich so zB lvars innerhalb von fresh auf eine lvar au\ss{}erhalb von fresh "ubetragen.

\item{\code{conde}:}
Mit conde ("ahnlich zu cond aus dem clojure.core Paket) lassen sich Constraints so gesagt \dq{}verodern\dq{}. Das heißt es erzeugt eine logische Disjunktion von Constraints.
\end{description}

Beispiel für \code{conde}:
\begin{lstlisting}
(run* [q]
    (conde
        ((unify q 2))
	  *OR*
        ((unify q 1) *AND* (unify q q))
	  *OR*
	((fresh [r s] 
	  (unify r 1)
	    *AND*
	  (unify s 2)
	    *AND*
	  (unify r q)
	    *AND*
	  (unify s q))
)
\end{lstlisting}

Das sind die 3 grundlegenden Funktionen von core.logic. Das gesamte Package beinhaltet aber nat"urlich noch viele mehr, Wie z.B. das eben gesehene (membero \dots{}). Alle weiteren Funktionen im Package bauen aber auf den 3 Basis Funktionen auf. H"ohere Funktionen, folgen einer bestimmten Namenskonvention, wie z.B. zu sehen bei conde und memebero, werden Funktionen in core.logic die schon im clojure.core existieren mit einem a,e,u oder o um diese von den regul"aren clojure Funktionen zu differenzieren und diese nicht zu "uberschreiben.

Ein nachgestelltes 
\begin{itemize}
  \item a steht f"ur 
  
  \item e steht f"ur 
  
  \item u steht f"ur 
  
  \item o steht f"ur die R"uckgabe von Goals bzw. einer Relation

\end{itemize}

\subsubsection{Aufruf des Solvers}

Einfaches Beispiel:

\begin{lstlisting}
( run 1 [q] 
	(== 1 q)
)
\end{lstlisting}

\begin{description}
\item{\code{run 1}:}
Mit \code{run} wird der Solver gestartet und dieser soll das erste Ergebnis, das er bekommt, zurückgeben.
\item{\code{[q]}:}
Das ist die logische Variable für die der Solver Werte suchen soll.
\item{\code{(== 1 q)}:}
Das ist die Beschränkung auf die logische Variable. q wird hier mit 1 unifiziert und gibt damit vor, dass q = 1 sein muss damit diese Beschränkung erfüllt ist.
\end{description}

Werden mehrere Beschränkungen definiert, macht es für den Solver keinen Unterschied in welcher Reihenfolge diese stehen. 

\subsubsection{Höhere Funktionen}

Neben den Funktionen \code{unify (==), fresh und conde} verfügt das Paket core.logic um einige weitere Funktionen die auf diesen 3 grundlegenden Funktionen aufbauen.

Dazu gehört zum Beispiel das bereits genannte \code{membero}:\\
\code{(membero x M)} beschränkt die logische Variable (in diesem Fall x) so, dass diese ein Element der Menge M sein muss, damit die Beschränkung erfüllt ist.\\
\begin{lstlisting}
(run * [q]
	(membero q [1 2 3])
)
\end{lstlisting}
Dieses Beispiel würde die Ausgabe \code{(1, 2, 3)} zurückgeben, da q eine dieser 3 Zahlen annehmen kann, um ein Element der Menge \code{[1 2 3]} zu sein.

\begin{lstlisting}
(defne membero
	[x l]
	([_ [x . tail]])
	([_ [head . tail]]
	 (membero x tail)
	)
)
\end{lstlisting}

Die Definition von \code{membero} in core.logic. Diese besteht aus zwei Beschränkungen, \code{([\_ [x . tail]])} und \code{([\_ [head . tail]] (membero x tail))}. Während die erste Beschränkung erfüllt ist, wenn x das erste Element der Menge l ist, besagt die zweite, dass wenn x nicht das erste Element der Menge ist, dann ist x das erste Element der Menge tail (wobei tail die Menge l ohne deren erstes Element darstellt).

Weitere höhere Funktionen sind:
\begin{description}
\item{\code{(resto l r)}:}
\code{resto} schränkt die logische Variable so ein, dass r die Restmenge der Menge l ist. Das heißt r ist die Menge l ohne deren erstes Element.
\begin{lstlisting}
(run * [q]
	(resto [1 2 3 4] q)
)
\end{lstlisting}
q entspricht in diesem Beispiel der Menge (2 3 4).
\item{\code{(conso x r s)}:}
\code{conso} schränkt die logische Variable so ein, dass x das erste Element einer Menge, r der Rest dieser Menge und s genau diese Menge ist.

\begin{lstlisting}
(run * [q]
	(conso 1 [2 3 4] q)
)
\end{lstlisting}
q entspricht hier der Menge (1 2 3 4)

\begin{lstlisting}
(run * [q]
	(conso q [2 3 4] [1 2 3 4])
)
\end{lstlisting}
q entspricht hier 1
\end{description}
